%%%%%%%%%%%%%%%%%%%%%%%%%%%%%%%%%%%%%%%%%%%%%%%%%%%%%%%%%%%%%%%%%%%%%%%%%%%%%%%%
%% Some custom packages
%%
\RequirePackage{xspace}

\usepackage{lmodern}
\usepackage{amssymb}
\usepackage{algorithm}
% \usepackage{algorithmicx}
\usepackage[noend]{algpseudocode}
\usepackage{booktabs}
\usepackage{graphicx} % draft
\usepackage{amsmath}
\usepackage{bm}
\usepackage{mathtools}
\usepackage{multicol}
\usepackage{tikz}
\usepackage{minted}
\usepackage{xcolor}
\usepackage{blindtext}
\usepackage{wrapfig}
\usepackage{pgfornament}
\usepackage[normalem]{ulem}
\usepackage{csquotes}

\usemintedstyle{pastie}
\usetikzlibrary{bayesnet}
\usetikzlibrary{positioning}

\usepackage[most]{tcolorbox}
\tcbuselibrary{breakable}

% Define the new environment
\newenvironment{casestudy}[1] % #1 is the parameter for the title
  {\begin{tcolorbox}[breakable,
                     enhanced,
                     colback=white,
                     colframe=black!80,
                     colbacktitle=black!80,
                     title=#1, % Use the title passed as a parameter
                     fonttitle=\bfseries\large,
                     coltitle=white,
                     boxed title style={sharp corners, colframe=black!80},
                     attach boxed title to top left={yshift=-2mm, xshift=2mm},
                     borderline={0.5mm}{0mm}{black!80},
                     sharp corners,
                     boxsep=5pt,
                     left=10pt,
                     right=10pt,
                     top=10pt,
                     bottom=10pt]
  }
  {\end{tcolorbox}}

\newenvironment{contrib}[1] % #1 is the title
  {\begin{tcolorbox}[breakable,
                     enhanced,
                     colback=white,
                     colframe=blue!50,
                     colbacktitle=blue!50,
                     title=#1,
                     fonttitle=\bfseries\large,
                     coltitle=white,
                     boxed title style={sharp corners, colframe=blue!50},
                     attach boxed title to top left={yshift=-2mm, xshift=2mm},
                     borderline={0.5mm}{0mm}{blue!50},
                     sharp corners,
                     boxsep=5pt,
                     left=10pt,
                     right=10pt,
                     top=10pt,
                     bottom=10pt]
  }
  {\end{tcolorbox}}

\newcommand{\mecite}[1]{\textbf{\textcolor{black!50}{\cite{#1}}}}

\definecolor{alertred}{HTML}{BF0603}

\newcommand{\comeback}[1]{\textcolor{alertred}{#1}}
\newcommand{\icomeback}[1]{\textcolor{alertred}{\textbf{#1}}}
\newcommand{\textcontrib}[1]{\textcolor{black!50}{#1}}

%%%%%%%%%%%%%%%%%%%%%%%%%%%%%%%%%%%%%%%%%%%%%%%%%%%%%%%%%%%%%%%%%%%%%%%%%%%%%%%%
%% Fonts (like different typewriter fonts etc.)
%%
%\RequirePackage[scaled=.87]{couriers}
%\RequirePackage[T1]{fontenc}
%\renewcommand\rmdefault{psb}




%%%%%%%%%%%%%%%%%%%%%%%%%%%%%%%%%%%%%%%%%%%%%%%%%%%%%%%%%%%%%%%%%%%%%%%%%%%%%%%%
%% Style (Changing the visual style of chapter headings and stuff.)
%%
\RequirePackage{titlesec}
% [Fixes issue #34 (see https://github.com/cambridge/thesis/issues/34). Solution from: http://tex.stackexchange.com/questions/299969/titlesec-loss-of-section-numbering-with-the-new-update-2016-03-15
\RequirePackage{etoolbox}
\makeatletter
\patchcmd{\ttlh@hang}{\parindent\z@}{\parindent\z@\leavevmode}{}{}
\patchcmd{\ttlh@hang}{\noindent}{}{}{}
\makeatother
% end of issue #34 fix]
\newcommand{\PreContentTitleFormat}{\titleformat{\chapter}[display]{\scshape\Large}
{\Large\filleft\MakeUppercase{\chaptertitlename} \Huge\thechapter}
{1ex}
{}
[\vspace{1ex}\titlerule]}
\newcommand{\ContentTitleFormat}{\titleformat{\chapter}[display]{\scshape\huge}
{\Large\filleft\MakeUppercase{\chaptertitlename} \Huge\thechapter}
{1ex}
{\titlerule\vspace{1ex}\filright}
[\vspace{1ex}\titlerule]}
\newcommand{\PostContentTitleFormat}{\PreContentTitleFormat}
\PreContentTitleFormat




%%%%%%%%%%%%%%%%%%%%%%%%%%%%%%%%%%%%%%%%%%%%%%%%%%%%%%%%%%%%%%%%%%%%%%%%%%%%%%%%
%% References (special style etc.)
%%

\RequirePackage[authoryear,round]{natbib}
\usepackage[hyperpageref]{backref}
\usepackage{cleveref}
\usepackage{hyperref}

\renewcommand{\backreftwosep}{ and~}
\renewcommand{\backreflastsep}{, and~}
\renewcommand*{\backref}[1]{}
\renewcommand*{\backrefalt}[4]{
    \ifcase #1
        (Not cited.)
    \or
        (Cited on page~#2.)
    \else
        (Cited on pages~#2.)
    \fi
}

%%%%%%%%%%%%%%%%%%%%%%%%%%%%%%%%%%

\usepackage{libertine}
\usepackage[libertine]{newtxmath}

\usepackage{setspace}
\setstretch{1.5}

\let\openbox\relax

\renewcommand{\ttdefault}{zi4}

%%%%%%%%%%%%%%%%%%%%%%%%%%%%%%%%%%%%%%%%%%%%%%%%%%%%%%%%%%%%%%%%%%%%%%%%%%%%%%%%
%% Theorems, definitions, and examples
%%
\RequirePackage{amsthm}
\theoremstyle{definition}
\newtheorem{definition}{Definition}[chapter]
%% Support for `Examples` (provides a counter for examples, the possibility to
%% label and reference them etc.)
%%
\newtheorem{theorem}{Theorem}[chapter]
\newtheorem{lemma}{Lemma}[chapter]
\newtheorem{example}{Example}[chapter]
\newtheorem{proposition}{Proposition}[chapter]

\DeclareMathOperator*{\argmin}{arg\,min}
\DeclareMathOperator*{\argmax}{arg\,max}

\let\Hungarian\H

\newcommand{\Y}{\mathbf{Y}}
\newcommand{\y}{\mathbf{y}}
\newcommand{\M}{\mathbf{M}}
\newcommand{\X}{\mathbf{X}}
\newcommand{\A}{\mathbf{A}}
\renewcommand{\L}{\mathbf{L}}
\newcommand{\I}{\mathbf{I}}
\renewcommand{\S}{\mathbf{S}}
\renewcommand{\H}{\mathbf{H}}
\newcommand{\T}{\mathbf{T}}
\newcommand{\C}{\mathbf{C}}
\newcommand{\U}{\mathbf{U}}
\renewcommand{\C}{\mathbf{C}}
\newcommand{\q}{{d_q}} % latent dim
\newcommand{\kt}{\Tilde{k}}
\renewcommand{\P}{\mathbf{P}}
\newcommand{\R}{\mathbb{R}}

\newcommand{\blu}{\textcolor{blue}}
\newcommand{\red}[1]{\textcolor{purple}{#1}}

\RequirePackage[small,bf]{caption}
\RequirePackage{subfig}

\graphicspath{{./figures/}}

%%%%%%%%%%%%%%%%%%%%%%%%%%%%%%%%%%%%%%%%%%%%%%%%%%%%%%%%%%%%%%%%%%%%%%%%%%%%%%%%
%% Glossary entries
%%
% Define a custom command for glossary entries
\newcommand{\newmathglossaryentry}[3]{%
  \newglossaryentry{#1}{%
    name={\ensuremath{#2}},%
    sort={#1},%
    description={#3}%
  }
}

\newmathglossaryentry{n}{n}{number of data points}
\newmathglossaryentry{d}{d}{number of data dimensions}
\newmathglossaryentry{dq}{\q}{number of latent dimensions}
\newmathglossaryentry{X}{\X}{\textbf{explanatory} variable set, often \textbf{latent}, in $\mathbb{R}^{n \times \q}$}
\newmathglossaryentry{Y}{\Y}{\textbf{response} data set, typically in $\mathbb R^{n \times d}$, with a row $\Y_{i}$ corresponding to a data point}
\newmathglossaryentry{sim}{X \sim \mathcal{D}}{a random variable $X$ is sampled from / follows a distribution $\mathcal D$}
\newmathglossaryentry{loss}{\mathcal{L}(\theta)}{a loss function, minimised w.r.t. $\theta$. Typically inversely related to a log-probability (or a lower bound on it) $\mathcal{L}=-\mathcal{E}$}
\newmathglossaryentry{energy}{\mathcal{E}(\theta)}{a log-likelihood, evidence lower bound or similar function, maximised w.r.t. $\theta$}
\newmathglossaryentry{prob-eval}{\log \mathcal{D}(x' | \theta, ...)}{the log probability (density) corresponding to a distribution $\mathcal D$ with parameters $\theta, ...$ evaluated at the value $x'$}
\newmathglossaryentry{Sh}{\hat{\S}}{an estimate of an empirical data-data covariance matrix, for example, calculated as $\Y\Y^\top/d$ with centered data $\Y$}
\newmathglossaryentry{Pr}{\hat{\boldsymbol{\Gamma}}}{an estimate of an empirical precision matrix, i.e. the inverse of a covariance matrix}
\newmathglossaryentry{A}{\A}{a symmetric adjacency matrix}
\newmathglossaryentry{L}{\L}{a graph Laplacian matrix}
\newmathglossaryentry{GL}{\text{GL}}{a graph Laplacian matrix}
\newmathglossaryentry{MN}{\M \sim \mathcal{MN}(\boldsymbol{\mu}, \boldsymbol{\Sigma}_r, \boldsymbol{\Sigma}_c)}{a matrix normal distribution with parameters $\boldsymbol{\mu}$, $\boldsymbol{\Sigma}_r$, $\boldsymbol{\Sigma}_c$. The following equivalence holds, $\text{vec}({\M}) \sim \mathcal{N}(\text{vec}({\boldsymbol{\mu}}), \boldsymbol{\Sigma}_c \otimes \boldsymbol{\Sigma}_r)$. In almost all cases in the thesis, we use the notation for summarising the following; if $\M_{:j}$ is the $j$-th column of $\M$ and has the following distribution independent to other random variables, $\forall j: \M_{:j} \sim \mathcal{N}(\boldsymbol{0}, \boldsymbol{\Sigma})$, then $\M \sim \mathcal{MN}(\boldsymbol{0}, \boldsymbol{\Sigma}, \I)$}
\newmathglossaryentry{sphere}{\mathcal{S}^{\q-1}}{unit hypersphere in $\q$ dimensions}
\newmathglossaryentry{vmf}{\text{vMF}}{von Mises-Fisher distribution}
\newmathglossaryentry{W}{\mathcal{W}}{a Wishart distribution with a parametrisation such that, if $\S \sim \mathcal{W}(\M, \nu)$, then, $\mathbb{E}(\S) = \nu\M$}
\newmathglossaryentry{Wi}{\mathcal{W}^{-1}}{an inverse-Wishart distribution with parametrisation such that, if $\S^{-1} \sim \mathcal{W}^{-1}(\M^{-1}, \nu)$, then, $\S \sim \mathcal{W}(\M, \nu)$}
\newmathglossaryentry{cen}{\text{centrality parameter}}{assuming an (inverse-)Wishart distribution $\mathbf{M} \sim \mathcal{W}^{\{-1\}}(\mathbf{C}, \nu)$, we denote $\mathbf{C}$ as the ``centrality'' parameter (instead of the scale parameter as is done traditionally), as $\mathbb E(\mathbf{M}) \propto \C$}
\newmathglossaryentry{I}{\I}{ identity matrix}
\newmathglossaryentry{H}{\H}{ centering matrix $\H = \I - \mathbf{11}^T/n$}
\newmathglossaryentry{pinv}{\M^+}{the Moore-Penrose pseudo-inverse of the matrix $\M$}
\newmathglossaryentry{Ic}{\mathcal{I}}{indicator function}
\newmathglossaryentry{p}{p}{a density function. We denote $p_\mathcal{M}$ to denote the density of a model $\mathcal{M}$ specified in the text, or a corresponding distribution, for example, $p_\mathcal{N}$ corresponding to the density of a normal distribution}
\newmathglossaryentry{probability}{\mathbb{P}}{a probability measure. We sometimes denote the probability $\mathbb P(X\in A)$ as $\mathbb P_X(A)$ or $\mathbb P(X)$ when the set A is unimportant for exposition}
\newmathglossaryentry{sigmoid}{\sigma(.)}{a sigmoid (inverse-logistic) or softmax function}
\newmathglossaryentry{elem}{\odot}{element-wise multiplication}
\newmathglossaryentry{kron}{\otimes}{Kronecker product}
\newmathglossaryentry{reim}{\Re, \Im}{real and imaginary parts}
\newmathglossaryentry{peq}{\overset+=}{$a\overset+=b \Rightarrow a = b + k$ for some uninteresting constant $k$}
\newmathglossaryentry{ul}{\mathbf{U}, \boldsymbol{\Lambda}}{typically used for Eigenvectors and Eigenvalues of a matrix}
\newmathglossaryentry{KL}{\mathrm{KL}(q\|p)}{Kullback---Leibler divergence between densities $q$ and $p$}
\newmathglossaryentry{kern}{k(\cdot, \cdot)}{typically a kernel function. We denote the matrix shorthand as, $\Sigma = K(\X, \X)$ meaning that $\Sigma_{ij}(\X) = k(\X_i, \X_j)$}
\newmathglossaryentry{psd}{\text{PSD}}{positive semi-definite}
\newmathglossaryentry{expm}{\text{expm}}{matrix exponential. $\text{exp}$ refers to an element-wise exponential}
\newmathglossaryentry{stop-grad}{\text{stop-grad}}{an operation $\text{stop-grad}[f(x)]$ sets the gradients of the argument $f(x)$, to zero}
\newmathglossaryentry{semantic}{\text{semantic}}{applies to model assumptions or characteristics, referring to the ``in words'' meaning of the models (as opposed to its formal mathematical description). For example, the ``in words'' meaning of a log-normal distributed variable could be that a variable is positive with its log being symmetrically distributed. By semantic consistency, we mean that two models correspond to equivalent ``in words'' readings}
\newmathglossaryentry{grammar}{\text{probabilistic grammar}}{probabilistic models forming grammars refers to the fact that many models, and probabilistic programming language statements translate to an ``in words'' description of a data characteristic being modelled}
\newmathglossaryentry{knn}{\text{kNN}}{k-nearest neighbour (graph)}
\newmathglossaryentry{major eigencomponents}{\text{major eigencomponents}}{the $\q$ major/principal eigenvectors and eigenvalues of a matrix $\S$ correspond to the largest $\q$ eigenvalues}
\newmathglossaryentry{minor eigencomponents}{\text{minor eigencomponents}}{the $\q$ minor eigenvectors and eigenvalues of a matrix $\S$ correspond to the smallest $\q$ eigenvalues}
\newmathglossaryentry{euc}{d^2(., .)}{the Euclidean distance metric, $d^2(\mathbf{x}, \mathbf{y}) = \| \mathbf{x} - \mathbf{y}\|^2$. We sometimes overload the notation by writing $d_{ij}^2(\X)$ to mean $d_{ij}^2(\X_i, \X_j)$}

\usepackage{etoolbox}

\newcommand{\dedicationpage}[1]{%
  \cleardoublepage
  \thispagestyle{empty}
  \vspace*{\fill}
  \begin{center}\emph{#1}\end{center}
  \vspace*{\fill}
  \cleardoublepage
}

% \makeatletter
% \pretocmd{\tableofcontents}{
%   \dedicationpage{For when one does not wish to abstract away a problem.}
% }{}{}
% \makeatother

